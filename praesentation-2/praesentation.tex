
\documentclass[authorinfoot]{fidius-p}
\usepackage{listings}
\lstset{language=Ruby}
% \usepackage{tabularx}
\author{AB \and JF}
% \title[short title]{long title}
% optional: \subtitle[short subtitle]{long subtitle}
\title{CVE-DB}
\subtitle{Präsentation}
\date{21.01.2011}

\setcounter{tocdepth}{1}

\setbeamertemplate{bibliography item}[text]

\setbeamertemplate{bibliography entry title}{}
\setbeamertemplate{bibliography entry location}{}
\setbeamertemplate{bibliography entry note}{}

\bibliographystyle{is-plain}

\begin{document}

\frame{\titlepage}
\frame{\tableofcontents}

\section{CVE-DB}
\secframe{Ziele}{
  \begin{itemize}
    \item Parsen der von der NVD bereitgestellten XML-Dateien und vorhalten in Ruby-Objekten. \checkmark
    \item Datenbankschema für CVE-Einträge in einem Rails Projekt erstellt.
\checkmark
    \item Ablegen der CVE-Einträge in einer Datenbank. \checkmark
    \item Abfrage der Datenbank über ActiveRecord möglich. \checkmark
    \item Rake-Tasks zur Bedienung des Parsers erstellt. \checkmark
    \item Umgang mit doppelten Datenbankeinträgen. \checkmark
    \item Integration in den C\&C-Server. (\checkmark)
    \item Zuordnung zwischen CVE- und MS-Notation. \checkmark
    \item Veröffentlichung auf Github.
  \end{itemize}
}

\subsection{Rake Tasks}
\begin{frame}[fragile]
  \frametitle{Rake Tasks}
  \begin{lstlisting}
rake nvd:get[xml_name]     # XML von nvd.org runterladen
rake nvd:initialize        # Initialisieren, XMLs einlesen
rake nvd:list_local        # Lokale XMLs anzeigen
rake nvd:list_remote       # Remote XMLs anzeigen
rake nvd:mscve             # Zuordnung CVE <-> MS Notation
rake nvd:parse[file_name]  # Datei parsen
rake nvd:update            # modified.xml laden, parsen
  \end{lstlisting}
\end{frame}


\secframe{fidius-cvedb GEM}{
  \begin{itemize}
    \item Enthält..
    \begin{itemize}
     \item Parser
     \item Models
     \item Migrations
     \item Rake-Tasks
    \end{itemize}
    \item 2 Initialisierungs-Möglichkeiten
    \begin{itemize}
     \item \textbf{''fidius''}
     \item Für Fidius-Kontext gedacht, CVE-DB existiert schon
     \item Erstellt nur Models, Hinweise zur DB-Konfiguration
    \end{itemize}
    \begin{itemize}
     \item \textbf{''standalone''}
     \item Für externe Nutzung gedacht, CVE-DB muss erzeugt werden
     \item Erstellt Models, Migrations, Parser und Rake-Tasks
     \item Hinweise zur DB-Konfiguration und zu benötigten GEMs.
    \end{itemize}
  \end{itemize}
}

\secframe{Suche}{
  \begin{itemize}
    \item CVE-DB enthält ``Products'', bestehend aus
    \begin{itemize}
     \item vendor - product - version - update\_nr - edition - language
     \item z.B: apache - http\_server - 1.3.11 - NULL - win32 - NULL
     \item z.B: microsoft - windows\_2003\_server - NULL - sp2 - itanium
    \end{itemize}
    \item Über Products lassen sich CVE-Nummern suchen
    \subitem{\texttt{has\_many :nvd\_entries, :through =>
vulnerable\_softwares}}
    \item Integration in C\&C-Server ist noch nicht abgeschlossen.
  \end{itemize}
}

\secframe{Retrospektive}{
  \begin{itemize}
    \item Etwas schleppender Start nach den Weihnachtsferien
    \item Gute Zusammenarbeit mit anderen Gruppen, Hilfe beim C\&C Server
    \item Tests werden etwas vor sich hergeschoben
  \end{itemize}
}

\secframe{Ausblick}{
  \begin{itemize}
    \item CVE-DB im Fidius-Kontext ist weitgehend abgeschlossen
    \begin{itemize}
     \item Für Standalone wäre noch eine eigene Anzeige nötig
     \item Vor Veröffentlichung brauchen wir noch Tests
     \item Suche sollte noch verbessert werden
     \subitem{ .. bzw. Scan Ergebnisse besser erfasst}
    \end{itemize}
    \item Abschlie"sen der Integration erfordert keine eigene AG
  \end{itemize}
}

\secframe{Fragen}{
Fragen / Anregungen / Wünsche ?
}

\end{document}

